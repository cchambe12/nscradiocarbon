\documentclass{article}\usepackage[]{graphicx}\usepackage[]{color}
% maxwidth is the original width if it is less than linewidth
% otherwise use linewidth (to make sure the graphics do not exceed the margin)
\makeatletter
\def\maxwidth{ %
  \ifdim\Gin@nat@width>\linewidth
    \linewidth
  \else
    \Gin@nat@width
  \fi
}
\makeatother

\definecolor{fgcolor}{rgb}{0.345, 0.345, 0.345}
\newcommand{\hlnum}[1]{\textcolor[rgb]{0.686,0.059,0.569}{#1}}%
\newcommand{\hlstr}[1]{\textcolor[rgb]{0.192,0.494,0.8}{#1}}%
\newcommand{\hlcom}[1]{\textcolor[rgb]{0.678,0.584,0.686}{\textit{#1}}}%
\newcommand{\hlopt}[1]{\textcolor[rgb]{0,0,0}{#1}}%
\newcommand{\hlstd}[1]{\textcolor[rgb]{0.345,0.345,0.345}{#1}}%
\newcommand{\hlkwa}[1]{\textcolor[rgb]{0.161,0.373,0.58}{\textbf{#1}}}%
\newcommand{\hlkwb}[1]{\textcolor[rgb]{0.69,0.353,0.396}{#1}}%
\newcommand{\hlkwc}[1]{\textcolor[rgb]{0.333,0.667,0.333}{#1}}%
\newcommand{\hlkwd}[1]{\textcolor[rgb]{0.737,0.353,0.396}{\textbf{#1}}}%
\let\hlipl\hlkwb

\usepackage{framed}
\makeatletter
\newenvironment{kframe}{%
 \def\at@end@of@kframe{}%
 \ifinner\ifhmode%
  \def\at@end@of@kframe{\end{minipage}}%
  \begin{minipage}{\columnwidth}%
 \fi\fi%
 \def\FrameCommand##1{\hskip\@totalleftmargin \hskip-\fboxsep
 \colorbox{shadecolor}{##1}\hskip-\fboxsep
     % There is no \\@totalrightmargin, so:
     \hskip-\linewidth \hskip-\@totalleftmargin \hskip\columnwidth}%
 \MakeFramed {\advance\hsize-\width
   \@totalleftmargin\z@ \linewidth\hsize
   \@setminipage}}%
 {\par\unskip\endMakeFramed%
 \at@end@of@kframe}
\makeatother

\definecolor{shadecolor}{rgb}{.97, .97, .97}
\definecolor{messagecolor}{rgb}{0, 0, 0}
\definecolor{warningcolor}{rgb}{1, 0, 1}
\definecolor{errorcolor}{rgb}{1, 0, 0}
\newenvironment{knitrout}{}{} % an empty environment to be redefined in TeX

\usepackage{alltt}
\usepackage{Sweave}
\usepackage{float}
\usepackage{subcaption}
\usepackage{graphicx}
\usepackage{tabularx}
\usepackage{siunitx}
\usepackage{multirow}
\usepackage{amssymb} % for math symbols
\usepackage{amsmath} % for aligning equations
%\usepackage{hyperref}
\usepackage{textcomp}
\usepackage{mdframed}
\usepackage{longtable}
\usepackage{natbib}
\bibliographystyle{..//bib/styles/gcb}
%\usepackage[hyphens]{url}
\usepackage{caption}
\setlength{\captionmargin}{30pt}
\setlength{\abovecaptionskip}{0pt}
\setlength{\belowcaptionskip}{10pt}
\topmargin -1.5cm        
\oddsidemargin -0.04cm   
\evensidemargin -0.04cm
\textwidth 16.59cm
\textheight 21.94cm 
%\pagestyle{empty} %comment if want page numbers
\parskip 7.2pt
\renewcommand{\baselinestretch}{1.5}
\parindent 0pt
\usepackage{lineno}
\linenumbers

\newmdenv[
  topline=true,
  bottomline=true,
  skipabove=\topsep,
  skipbelow=\topsep
]{siderules}

%% R Script


\IfFileExists{upquote.sty}{\usepackage{upquote}}{}
\begin{document}
\noindent 
\renewcommand{\thetable}{S\arabic{table}}
\renewcommand{\thefigure}{S\arabic{figure}}

\textbf{\LARGE{Methods: Data analysis}}

\par We estimated the effects of increment depth (i.e., 0-1cm, 1-2cm, 2-3cm, 3-4cm, 4-8cm, 8cm-pith) on total, sugar and starch concentrations for both ring-porous and diffuse-porous species using Bayesian hierarchical models with the brms package \citep{brms}, version 2.3.1,  in R \citep{R}, version 3.3.1. Seasons are modeled hierarchically, generating an estimate of the overall response across season and estimates of season-level responses---and the distribution from which they are drawn. The intercept in the model is increment 0-1cm since the predictors are discrete groups:  

\begin{align*}
y_i &= \alpha_{season[i]} + \beta_{increment1-2_{season[i]}} + \beta_{increment2-3_{season[i]}} + \beta_{increment3-4_{season[i]}}\\
&+ \beta_{increment4-8_{season[i]}} + \beta_{increment8-pith_{season[i]}} + \epsilon_i\\,
\end{align*}
\begin{align*}
\epsilon_i & \sim N(0,\sigma^2_y) \\
\end{align*}
\noindent The $\alpha$ and each of the five $\beta$ coefficients were modeled at the season level, as follows:
\begin{align*}
\alpha_{season} & \sim N(\mu_{\alpha}, \sigma_{\alpha}) \\
\beta_{increment1-2_{season}} & \sim N(\mu_{increment1-2}, \sigma_{increment1-2}) \\
\beta_{increment2-3_{season}} & \sim N(\mu_{increment2-3}, \sigma_{increment2-3}) \\
\beta_{increment3-4_{season}} & \sim N(\mu_{increment3-4}, \sigma_{increment3-4}) \\
\beta_{increment4-8_{season}} & \sim N(\mu_{increment4-8}, \sigma_{increment4-8}) \\
\beta_{increment8-pith_{season}} & \sim N(\mu_{increment8-pith}, \sigma_{increment8-pith}) \\
\end{align*}

where $i$ represents each unique observation, $season$ is the season, $\alpha$ represents the intercept, which is again increment 0-1cm, $\beta$ terms represent slope estimates, and $y$ is the total, sugar or starch concentration. 

We additionally estimated the effects of organ on the radiocarbon age using a Bayesian model \citep{brms, R}, where branch is the intercept since, again, the predictors are discrete groups:

\begin{align*}
y_i &= \alpha_{[i]} + \beta_{shoot0-0.5_{[i]}} + \beta_{shoot0.5-1_{[i]}} + \beta_{shoot1-1.5_{[i]}}\\
&+ \beta_{shoot1.5-2_{[i]}} + \beta_{root0-0.5_{[i]}} + \beta_{root0.5-1_{[i]}} + \beta_{root1-1.5_{[i]}}\\
&+ + \beta_{root1.5-2_{[i]}} + \beta_{fineroot_{[i]}} +\epsilon_{[i]}\\,
\end{align*}
\begin{align*}
\epsilon_i & \sim N(0,\sigma^2_y) \\
\end{align*}

We ran four chains, each with 2,500 warm-up iterations and 4,000 sampling iterations for a total of 6,000 posterior samples for each predictor for each model. We evaluated our model performance based on $\hat{R}$ values that were close to one. We also evaluated high $n_{eff}$ (4000 for most parameters, but as low as 794 for a few parameters in the diffuse-porous sugar concentration model). We additionally assessed chain convergence and posterior predictive checks visually \citep{BDA}.

\bibliography{..//bib/nsc.bib}

\textbf{\LARGE{Supplement: Tables}} 

% latex table generated in R 3.6.0 by xtable 1.8-4 package
% Mon Nov 18 16:04:02 2019
\begin{table}[ht]
\centering
\caption{\textbf{Estimates from ring-porous total concentration (mg/g) model}. Using a model testing the effects of increment depth on total concentration (mg/g) for ring-porous species, results in slightly muted variation in concentration across increments. We present posterior means, as well as 50 percent and 97.5 percent uncertainty intervals from the model.} 
\label{tab:ringtot}
\begingroup\footnotesize
\begin{tabular}{|p{0.14\textwidth}|p{0.06\textwidth}|p{0.06\textwidth}|p{0.06\textwidth}|p{0.06\textwidth}|p{0.06\textwidth}|p{0.06\textwidth}|}
  \hline
 & mean & sd & 25\% & 75\% & 2.5\% & 97.5\% \\ 
  \hline
$\mu_{\alpha}$ & 48.75 & 5.64 & 45.38 & 52.41 & 34.61 & 61.61 \\ 
  $\mu_{increment 1-2}$ & -22.00 & 6.59 & -25.68 & -18.23 & -38.50 & -6.22 \\ 
  $\mu_{increment 2-3}$ & -32.26 & 6.93 & -36.05 & -28.16 & -49.72 & -16.45 \\ 
  $\mu_{increment 3-4}$ & -32.46 & 7.10 & -36.46 & -28.21 & -50.44 & -15.62 \\ 
  $\mu_{increment 4-8}$ & -34.67 & 6.80 & -38.55 & -30.63 & -51.45 & -18.08 \\ 
  $\mu_{increment 8-pith}$ & -36.18 & 11.51 & -41.00 & -30.50 & -64.36 & -10.25 \\ 
   \hline
\end{tabular}
\endgroup
\end{table}


% latex table generated in R 3.6.0 by xtable 1.8-4 package
% Mon Nov 18 16:04:02 2019
\begin{table}[ht]
\centering
\caption{\textbf{Estimates from diffuse-porous total concentration (mg/g) model}. Using a model testing the effects of increment depth on total concentration (mg/g) for diffuse-porous species, results in slightly greater variation in concentration across increments. We present posterior means, as well as 50 percent and 97.5 percent uncertainty intervals from the model.} 
\label{tab:difftot}
\begingroup\footnotesize
\begin{tabular}{|p{0.14\textwidth}|p{0.06\textwidth}|p{0.06\textwidth}|p{0.06\textwidth}|p{0.06\textwidth}|p{0.06\textwidth}|p{0.06\textwidth}|}
  \hline
 & mean & sd & 25\% & 75\% & 2.5\% & 97.5\% \\ 
  \hline
$\mu_{\alpha}$ & 31.77 & 6.84 & 27.51 & 36.08 & 14.97 & 47.38 \\ 
  $\mu_{increment 1-2}$ & -10.57 & 6.99 & -14.66 & -6.45 & -27.74 & 6.26 \\ 
  $\mu_{increment 2-3}$ & -17.99 & 8.67 & -23.12 & -12.86 & -38.27 & 3.54 \\ 
  $\mu_{increment 3-4}$ & -18.54 & 8.65 & -23.66 & -13.33 & -39.66 & 1.07 \\ 
  $\mu_{increment 4-8}$ & -20.45 & 8.91 & -25.63 & -14.98 & -43.56 & 0.38 \\ 
  $\mu_{increment 8-pith}$ & -17.73 & 10.07 & -22.31 & -12.79 & -44.28 & 8.27 \\ 
   \hline
\end{tabular}
\endgroup
\end{table}


% latex table generated in R 3.6.0 by xtable 1.8-4 package
% Mon Nov 18 16:04:02 2019
\begin{table}[ht]
\centering
\caption{\textbf{Estimates from ring-porous sugar concentration (mg/g) model}. Using a model testing the effects of increment depth on sugar concentration (mg/g) for ring-porous species, results in slightly muted variation in concentration across increments. We present posterior means, as well as 50 percent and 97.5 percent uncertainty intervals from the model.} 
\label{tab:ringsug}
\begingroup\footnotesize
\begin{tabular}{|p{0.14\textwidth}|p{0.06\textwidth}|p{0.06\textwidth}|p{0.06\textwidth}|p{0.06\textwidth}|p{0.06\textwidth}|p{0.06\textwidth}|}
  \hline
 & mean & sd & 25\% & 75\% & 2.5\% & 97.5\% \\ 
  \hline
$\mu_{\alpha}$ & 31.47 & 5.00 & 28.43 & 34.53 & 19.25 & 42.70 \\ 
  $\mu_{increment 1-2}$ & -11.69 & 5.62 & -14.83 & -8.66 & -25.07 & 2.17 \\ 
  $\mu_{increment 2-3}$ & -15.01 & 5.56 & -18.21 & -11.74 & -28.84 & -1.77 \\ 
  $\mu_{increment 3-4}$ & -15.99 & 5.28 & -18.96 & -12.97 & -29.23 & -2.87 \\ 
  $\mu_{increment 4-8}$ & -17.32 & 5.63 & -20.50 & -14.00 & -31.54 & -4.06 \\ 
  $\mu_{increment 8-pith}$ & -19.12 & 11.56 & -24.96 & -13.17 & -49.91 & 11.69 \\ 
   \hline
\end{tabular}
\endgroup
\end{table}


% latex table generated in R 3.6.0 by xtable 1.8-4 package
% Mon Nov 18 16:04:02 2019
\begin{table}[ht]
\centering
\caption{\textbf{Estimates from diffuse-porous sugar concentration (mg/g) model}. Using a model testing the effects of increment depth on sugar concentration (mg/g) for diffuse-porous species, results in slightly greater variation in concentration across increments. We present posterior means, as well as 50 percent and 97.5 percent uncertainty intervals from the model.} 
\label{tab:diffsug}
\begingroup\footnotesize
\begin{tabular}{|p{0.14\textwidth}|p{0.06\textwidth}|p{0.06\textwidth}|p{0.06\textwidth}|p{0.06\textwidth}|p{0.06\textwidth}|p{0.06\textwidth}|}
  \hline
 & mean & sd & 25\% & 75\% & 2.5\% & 97.5\% \\ 
  \hline
$\mu_{\alpha}$ & 14.59 & 4.05 & 12.39 & 16.92 & 4.50 & 24.33 \\ 
  $\mu_{increment 1-2}$ & -1.38 & 1.38 & -2.06 & -0.70 & -4.98 & 2.22 \\ 
  $\mu_{increment 2-3}$ & -1.85 & 1.27 & -2.52 & -1.21 & -5.03 & 1.49 \\ 
  $\mu_{increment 3-4}$ & -2.58 & 1.68 & -3.33 & -1.83 & -7.07 & 1.67 \\ 
  $\mu_{increment 4-8}$ & -3.14 & 2.80 & -4.58 & -1.68 & -10.74 & 4.21 \\ 
  $\mu_{increment 8-pith}$ & -6.82 & 12.60 & -11.04 & -1.85 & -38.82 & 18.67 \\ 
   \hline
\end{tabular}
\endgroup
\end{table}


% latex table generated in R 3.6.0 by xtable 1.8-4 package
% Mon Nov 18 16:04:02 2019
\begin{table}[ht]
\centering
\caption{\textbf{Estimates from ring-porous starch concentration (mg/g) model}. Using a model testing the effects of increment depth on starch concentration (mg/g) for ring-porous species, results in slightly muted variation in concentration across increments. We present posterior means, as well as 50 percent and 97.5 percent uncertainty intervals from the model.} 
\label{tab:ringstar}
\begingroup\footnotesize
\begin{tabular}{|p{0.14\textwidth}|p{0.06\textwidth}|p{0.06\textwidth}|p{0.06\textwidth}|p{0.06\textwidth}|p{0.06\textwidth}|p{0.06\textwidth}|}
  \hline
 & mean & sd & 25\% & 75\% & 2.5\% & 97.5\% \\ 
  \hline
$\mu_{\alpha}$ & 16.96 & 5.06 & 13.94 & 20.14 & 4.48 & 28.54 \\ 
  $\mu_{increment 1-2}$ & -10.34 & 4.92 & -12.97 & -7.46 & -24.04 & 1.31 \\ 
  $\mu_{increment 2-3}$ & -17.40 & 6.09 & -20.80 & -13.97 & -32.41 & -2.30 \\ 
  $\mu_{increment 3-4}$ & -16.70 & 6.20 & -20.15 & -13.14 & -32.41 & -1.54 \\ 
  $\mu_{increment 4-8}$ & -17.54 & 5.87 & -20.91 & -14.17 & -31.57 & -2.65 \\ 
  $\mu_{increment 8-pith}$ & -14.26 & 8.43 & -18.01 & -10.33 & -36.81 & 7.00 \\ 
   \hline
\end{tabular}
\endgroup
\end{table}


% latex table generated in R 3.6.0 by xtable 1.8-4 package
% Mon Nov 18 16:04:02 2019
\begin{table}[ht]
\centering
\caption{\textbf{Estimates from diffuse-porous starch concentration (mg/g) model}. Using a model testing the effects of increment depth on starch concentration (mg/g) for diffuse-porous species, results in slightly greater variation in concentration across increments. We present posterior means, as well as 50 percent and 97.5 percent uncertainty intervals from the model.} 
\label{tab:diffstar}
\begingroup\footnotesize
\begin{tabular}{|p{0.14\textwidth}|p{0.06\textwidth}|p{0.06\textwidth}|p{0.06\textwidth}|p{0.06\textwidth}|p{0.06\textwidth}|p{0.06\textwidth}|}
  \hline
 & mean & sd & 25\% & 75\% & 2.5\% & 97.5\% \\ 
  \hline
$\mu_{\alpha}$ & 17.02 & 6.37 & 13.02 & 21.15 & 1.08 & 31.40 \\ 
  $\mu_{increment 1-2}$ & -9.06 & 7.27 & -13.30 & -4.65 & -27.85 & 8.11 \\ 
  $\mu_{increment 2-3}$ & -15.90 & 8.01 & -20.79 & -10.71 & -35.93 & 2.10 \\ 
  $\mu_{increment 3-4}$ & -15.81 & 8.46 & -20.82 & -10.68 & -35.78 & 4.82 \\ 
  $\mu_{increment 4-8}$ & -17.19 & 8.15 & -21.93 & -12.45 & -37.04 & 1.88 \\ 
  $\mu_{increment 8-pith}$ & -9.99 & 7.63 & -13.97 & -6.08 & -30.27 & 9.98 \\ 
   \hline
\end{tabular}
\endgroup
\end{table}


% latex table generated in R 3.6.0 by xtable 1.8-4 package
% Mon Nov 18 16:04:02 2019
\begin{table}[ht]
\centering
\caption{\textbf{Estimates from ring-porous radiocarbon ages}. Using a model testing the effects of organ on radiocarbon age for ring-porous species, results in muted variation in age across organs. We present posterior means, as well as 50 percent and 97.5 percent uncertainty intervals from the model.} 
\label{tab:ringradio}
\begingroup\footnotesize
\begin{tabular}{|p{0.14\textwidth}|p{0.06\textwidth}|p{0.06\textwidth}|p{0.06\textwidth}|p{0.06\textwidth}|p{0.06\textwidth}|}
  \hline
 & mean & 25\% & 75\% & 2.5\% & 97.5\% \\ 
  \hline
$\mu_{\alpha}$ & 1.37 & -0.43 & 3.13 & -4.98 & 7.53 \\ 
  $\mu_{shoot 0-0.5}$ & -0.42 & -2.95 & 2.24 & -9.57 & 8.01 \\ 
  $\mu_{shoot 0.5-1}$ & 3.47 & 0.99 & 5.96 & -5.27 & 12.07 \\ 
  $\mu_{shoot 1-1.5}$ & 3.46 & 0.90 & 6.02 & -5.44 & 12.01 \\ 
  $\mu_{shoot 1.5-2}$ & 7.56 & 5.03 & 9.98 & -0.86 & 16.29 \\ 
  $\mu_{root 0-0.5}$ & 1.27 & -1.29 & 3.86 & -7.44 & 9.75 \\ 
  $\mu_{root 0.5-1}$ & 4.04 & 1.53 & 6.62 & -4.56 & 12.45 \\ 
  $\mu_{root 1-1.5}$ & 7.03 & 4.54 & 9.59 & -1.53 & 15.50 \\ 
  $\mu_{root 1.5-2}$ & 9.40 & 6.95 & 11.93 & 0.61 & 17.96 \\ 
  $\mu_{fineroot}$ & 2.91 & 0.36 & 5.47 & -6.04 & 11.63 \\ 
   \hline
\end{tabular}
\endgroup
\end{table}


% latex table generated in R 3.6.0 by xtable 1.8-4 package
% Mon Nov 18 16:04:02 2019
\begin{table}[ht]
\centering
\caption{\textbf{Estimates from diffuse-porous radiocarbon ages}. Using a model testing the effects of organ on radiocarbon age for diffuse-porous species, results in slight greater variation in age across organs. We present posterior means, as well as 50 percent and 97.5 percent uncertainty intervals from the model.} 
\label{tab:diffradio}
\begingroup\footnotesize
\begin{tabular}{|p{0.14\textwidth}|p{0.06\textwidth}|p{0.06\textwidth}|p{0.06\textwidth}|p{0.06\textwidth}|p{0.06\textwidth}|}
  \hline
 & mean & 25\% & 75\% & 2.5\% & 97.5\% \\ 
  \hline
$\mu_{\alpha}$ & 2.24 & 0.95 & 3.51 & -2.07 & 6.55 \\ 
  $\mu_{shoot 0-0.5}$ & -0.62 & -2.42 & 1.12 & -6.63 & 5.66 \\ 
  $\mu_{shoot 0.5-1}$ & 2.39 & 0.59 & 4.23 & -3.73 & 8.29 \\ 
  $\mu_{shoot 1-1.5}$ & 5.18 & 3.37 & 6.97 & -0.88 & 11.09 \\ 
  $\mu_{shoot 1.5-2}$ & 7.04 & 5.27 & 8.82 & 0.91 & 13.17 \\ 
  $\mu_{root 0-0.5}$ & -1.29 & -3.10 & 0.52 & -7.13 & 4.83 \\ 
  $\mu_{root 0.5-1}$ & 1.30 & -0.47 & 3.02 & -4.80 & 7.33 \\ 
  $\mu_{root 1-1.5}$ & 4.15 & 2.33 & 5.93 & -1.89 & 10.34 \\ 
  $\mu_{root 1.5-2}$ & 6.18 & 4.38 & 7.97 & -0.21 & 12.70 \\ 
  $\mu_{fineroot}$ & 9.07 & 7.33 & 10.89 & 2.99 & 14.91 \\ 
   \hline
\end{tabular}
\endgroup
\end{table}


\end{document}
